\chapter{Introduzione alla programmazione lineare a numeri interi}

\section{Stato attuale delle wireless nerwork}
Oggigiorno non esistono pi� dispositivi isolati che non comunicano con altri,
 tutti i PC prevedono la connessione ad internet, o in generale comunicano con altri dispositivi elettronici
grazie alle reti di comunicazione. Inizialmente queste reti erano basate sui cavi ({\em wired network}),
 poi si sono rivelate pi� pratiche
le reti senza filo ({\em wireless network}); da ci� si � dovuto affrontare il problema
 di permettere l'utilizzo delle tecnologie
attualmente utilizzate nelle reti wired anche in reti wireless, che per loro natura sono intrinsecamente meno sicure
a causa del mezzo trasmissivo.
Le informazioni non vengono pi� convertite sotto forma di impulso elettronico e spedite su un filo metallico,
ma essendo l'etere il mezzo di propagazione sono convertite in onde radio; quest'ultime hanno una portata limitata
non sempre ben determinabile, sono soggette a disturbi che degradano le informazioni inviate fino, a volte, farle perdere;
ed infine cosa a noi maggiormente sgradita l'energia consumata e direttamente connessa alla potenza trasmessa,
ci� significa che aumentare la portata di trasmissione e la qualit� del segnale comporta un consumo
 maggiore dell'energia consumata.



\section{Dinamic Power Management}
Essendo la parte a radio frequenza colpevole del maggior consumo di energia, in particolare gli amplificatori
utilizzati immediatamente prima dell'invio del segnale o dopo una ricezione, le stazioni 802.11 possono allungare
la vita delle batterie spegnendo i transceiver radio e mettendo la schada wireless in sleep periodicamente.

Nella figura sottostante (Figura~\ref{consumosenzaps}), si pu� osservare una finestra di 20(CONTROLLLARE!!!) secondi del consumo di energia
di una scheda wireless senza il power management abilitato:
\begin{figure}[htbp]
%\centerline{\psfig{file=Misura1.eps, width = \columnwidth}}
\caption{Consumo di energia senza power management}
\label{consumosenzaps}
\end{figure}

Attivando il power management si pu� notare un sensibile risparmio energetico, a fronte di un diverso consumo
energetico dovuto ad un diverso modo di operare: 
\begin{figure}[htbp]
\begin{center}
%%\includegraphics{Misua1.eps}
\end{center}
\caption{Consumo di energia con power management}
\label{FUGURA CONSUMO POTENZA CON PS}
\end{figure}
in questa modalit� la stazione resta per la maggior parte del tempo nello stato di sleep,
risvegliandosi periodicamente (nella figura ogni 100ms) per controllare
 se nel frattempo c'� stato traffico per lei.
Durante il perido di sleep, sar� demandato all'Access Point il compito di bufferizzare i frame destinati verso
la scheda momentaneamente in sleep; al risveglio i frame bufferizzati saranno annunciati da un
{\em Beacon frame}, la ricezione da parte della stazione wireless
dei frame bufferizzati inizier� subito dopo con la loro richiesta tramite il {\em PS-Poll frame}.
(Figura~\ref{powersaving}).
\begin{figure}[htbp]
%%\centerline{\psfig{file=Misura4.eps, width = \columnwidth}}
\caption{Visione beacon e Ps-Poll}
\label{powersaving}
\end{figure}
Quando viene attivato il power management l'Access Point e la stazione wireless devono sincronizzarsi
per risvegliarsi ad intervalli prestabiliti, che nel caso della figura soprastante � 100ms;
all'inizio di ogni risveglio l'Access Point invier� il beacon contenete la {\em traffic indication map - TIM}
che dice alla stazione se vi sono pacchetti bufferizzati destinati a lei;
in tal cosa la stazione risponder�
con un PS-Pool contenente il suo Association ID.
Grazie all'Association ID l'Access Point sar� in grado di selezionare i frame destinati alla stazione
e provveder� ad inviarli.