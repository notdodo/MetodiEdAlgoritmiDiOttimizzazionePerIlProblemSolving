\chapter{Modelli e formulazioni matematiche}

\section{The Traveling Salesman Problem}
Il Traveling Salesman Problem (TSP) � il problema pi� noto dell'ottimizzazione combinatoria.
Siano date \emph{n} citt� e i costi $c_{ij}$ per andare dalla citt� \emph{i} alla citt� \emph{j}.
Si vuole determinare un cammino che parte da una citt� (diciamo $i_{1}$), visitare una ed una sola volta tutte le rimanenti citt� e terminare nella citt� di partenza $i_{1}$.
Inoltre si vuole che il costo di tale cammino sia minimo.\newline
Ha molteplici applicazioni pratiche e teoriche perche � la struttura di molti problemi pratici.\newline
Si � soliti modella il TSP come segue:
\begin{itemize}
	\item � dato un grafo orientato (o non orientato) \emph{G = (N,A)}
	dove \emph{N} � un insieme di \emph{n} vertici e \emph{A} � un insieme di \emph{m} archi.
	
	Ad ogni arco $\emph{(i,j)} \in \emph{A}$ � associato un costo $c_{ij}$.
	
	Un circuito hamiltoniano di \emph{G} � un circuito che passa per ogni vertice una ed una sola volta.\newline
	Il costo di un circuito hamiltoniano di \emph{G} � pari alla somma dei costi degli archi che compongono il circuito;
	\item il problema del TSP � di trovare un grafo \emph{G}, con una data matrice dei costi \emph{[$c_{ij}$]}, un circuito hamiltoniano di costo minimo.
\end{itemize}

\section{Formulazioni Matematiche del TSP}
In letteratura esistono molteplici (e a volte fantasiose) formulazioni del TSP.\newline
Presentiamo le due formulazioni pi� note e su cui si basano i metodi esatti pi� efficienti.

\subsection{TSP asimmetrico}
I costi $c_{ij}$ non verificano $c_{ij} = c_{ji}\;\forall\;i,j$ con $i < j$.\newline
Sia $x_{ij}$ una variabile $(0-1)$ associata ad ogni arco $(i,j) \in A$ dove $x_{ij}=1$ se l'arco $(i,j)$ � nella soluzione ottima e $x_{ij}=0$ altrimenti.\newline

\begin{equation}
	Min\;\;\displaystyle\sum_{i\in N}^{} \sum_{j\in N}^{} c_{ij} x_{ij}
\end{equation}
\begin{equation}
	s.t.\;\;\displaystyle\sum_{i\in N}^{} x_{ij} = 1,\;\;\forall j \in N
\end{equation}
\begin{equation}
	\;\;\;\;\;\;\;\displaystyle\sum_{j\in N}^{} x_{ij} = 1,\;\;\forall i \in N
\end{equation}
\begin{equation}
	\;\;\;\;\;\;\;\;\;\;\;\;\;\;\;\;\;\;\displaystyle\sum_{i\in S}^{} \sum_{j\in N\setminus S}^{} x_{ij} \ge 1,\;\;\forall S \subset N
\end{equation}
\begin{equation}
	\;\;\;\;\;\;\;\;\;\;x_{ij} \in \{0,1\}\;,\;\forall (i,j) \in A
\end{equation}

Il vincolo \emph{1.4} impone che ogni soluzione ammissibile debba contenere almeno un arco \emph{(i,j)} con $i\in S$ e $j\in N\setminus S$ per ogni sottoinsieme \emph{S} di \emph{N}.
Un'alternativa al vincolo \emph{1.4} �:
\begin{equation}\tag{1.4'}
\;\;\;\;\;\;\;\;\;\;\;\;\;\;\;\;\;\;\displaystyle\sum_{i\in S}^{} \sum_{j\in S}^{} x_{ij} \le |S| - 1,\;\;\forall S \subset N
\end{equation}

\subsection{TSP simmetrico}
Sia dato un grafo non-orientato \emph{G=(N,A)} con $c_{ij} = c_{ji}\:,\forall i,j\in N$.\newline
Gli archi di \emph{A} sono numerati da $1$ a $m$. L'arco di indice $l$ corrisponde a $(\alpha_{l},\beta_{l})$ con $\alpha_{l} < \beta_{l}$.\newline
$A_{i}$ � il sottoinsieme degli indici degli archi che incidono sul vertice $i$:
\begin{center}
	$A_{i} = \{l:\;l=1,m\;\;s.t.\;\alpha_{l}=i\;or\;\beta_{l}=i\}$
\end{center}

Per una dato $S\in N$ e $\bar{S} = N\setminus S$ indichiamo con $(S, \bar{S})$ il sottoinsieme degli indici degli archi per cui $\alpha_{l}\in S$ e $\beta_{l}\in \bar{S}$ oppure $\alpha_{l}\in \bar{S}$ e $\beta_{l}\in S$.

Ad ogni arco di incide $l$ � associato un costo $d_{l}=c_{\alpha_{l}\beta_{l}}$ e $x_{l}\in \{0,1\}$ � una variabile che vale 1 se e solo se l'arco di indice $l$ � nella soluzione ottima.
\begin{equation}
	Min\;\;\displaystyle\sum_{l=1}^{} d_{l}\:x_{l}
\end{equation}
\begin{equation}
	s.t.\;\;\displaystyle\sum_{l\in A_{i}}^{} x_{l}=2,\; \forall i\in N
\end{equation}
\begin{equation}
	\;\;\;\;\;\;\;\;\;\displaystyle\sum_{l\in (S, \bar{S})}^{} x_{l} \ge 1,\;\forall S \subset N
\end{equation}
\begin{equation}
	x_{l}\in \{0,1\},\;\;l=1,...,m
\end{equation}

\subsection{Eliminiazione subtours di Miller, Tucker, Zemlin (1960)}
Sia $u_{i}$ una variabile intera il cui valore sappresenta la posizione che il vertice $i$ occupa nel tour.

\begin{center}
	Es. tour (1,4,5,3,2,1) per TSP con n=5 vertici, si ha $u_{1}=1,\;u_{2}=5,\;u_{3}=4,\;u_{4}=2,\;u_{5}=3$	
\end{center}

Miller, Tucker e Zemlin propongono in alternativa a:
\begin{equation}\tag{*}
	\displaystyle\sum_{i\in S}^{}\sum_{j\in N\setminus S}^{} x_{ij} \ge 1,\;\;\forall\;S\subset N
\end{equation}